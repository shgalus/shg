\documentclass[a4paper,11pt]{article}
\usepackage{polski}
\usepackage[utf8]{inputenc}
\usepackage{amsmath}
\usepackage{amsthm}
\usepackage{amssymb}
\usepackage[unicode, colorlinks, linktocpage, linkcolor = blue, urlcolor =
  blue] {hyperref}

\newcommand{\naturaln}{\ensuremath{\mathbb{N}}}
\newcommand{\naturalnz}{\ensuremath{\mathbb{N}_0}}
\newcommand{\integern}{\ensuremath{\mathbb{Z}}}
\newcommand{\rationaln}{\ensuremath{\mathbb{Q}}}
\newcommand{\realn}{\ensuremath{\mathbb{R}}}
\newcommand{\complexn}{\ensuremath{\mathbb{C}}}
\renewcommand{\qedsymbol}{$\blacksquare$}

\theoremstyle{definition}
\newtheorem{note}{Note}

\title{Notatki z algebry}
\author{Stanisław Galus}
\date{\today}

\begin{document}
\begingroup
\renewcommand{\thepage}{0}
\maketitle
\endgroup
\tableofcontents

\begin{note}[8 V 2024] \label{note:fmvrv}
  Pierścień $\integern[x] / (x^n) / (a + bx)$, $n \in \naturaln$, $a, b \in \integern$.

  Stosuję oznaczenie
\end{note}

\bibliographystyle{plain}
\bibliography{shg}
\end{document}
