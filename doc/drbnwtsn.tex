\chapter{Cumulative distribution function of the Durbin-Watson statistic}

Let us consider the linear regression model \[ y_i = \beta_0 +
\beta_1x_{i1} + \ldots + \beta_kx_{ik} + \xi_i, \] $i = 1, \ldots, n$,
$k \geq 1$, $n > k + 1$, where $\xi_i$ are independent identically
distributed normal variables with zero mean, or $y = X\beta + \xi$. The
least squares estimate of $\beta$ is $b = (X^TX)^{-1}X^Ty$ and the
vector of residuals from the regression is $e = y - Xb$. The
Durbin-Watson $d$ statistic is defined by \[ d = \frac{\sum_{i = 2}^n
  (e_i - e_{i - 1})^2}{\sum_{i = 1}^n e_i^2} \] It is shown in
\cite[p.~426]{durbin-watson-1950} that $d_L \leq d \leq d_U$, where \[
d_L = \frac{\sum_{i = 1}^{n - k - 1} \lambda_i \zeta_i^2} {\sum_{i =
    1}^{n - k - 1} \zeta_i^2}, \] \[ d_U = \frac{\sum_{i = 1}^{n - k -
    1} \lambda_{i + k} \zeta_i^2} {\sum_{i = 1}^{n - k - 1}
  \zeta_i^2}, \] \[ \lambda_i = 2\left(1 - \cos \frac{\pi i}{n}\right),
\quad i = 1, \ldots, n - 1, \] and $\zeta_1, \ldots, \zeta_{n - 1}$ are
independent standard normal variables. The cumulative distribution
functions of $d_L$ and $d_U$ are
\begin{align} \label{eq:dwcdfdL}
  \Pr(d_L < x) &= \Pr\left(\sum_{i = 1}^{n - k - 1} (x - \lambda_i)
  \zeta_i^2 > 0 \right), \\
  \label{eq:dwcdfdU}
  \Pr(d_U < x) &= \Pr\left(\sum_{i = 1}^{n - k - 1} (x - \lambda_{i + k})
  \zeta_i^2 > 0 \right).
\end{align}

In both cases, the probabilities~(\ref{eq:dwcdfdL})
and~(\ref{eq:dwcdfdU}) are of the form \[ \Pr\left(\sum_{i = 1}^m w_i
\zeta_i^2 > 0 \right), \] where $w_i$, $i = 1, \ldots, m$, $m > 0$, are
those $x - \lambda_i$ or $x - \lambda_{i + k}$ which are not equal to 0.
They can be calculated by the formula (3.2) of
\cite[p.~422]{imhof-1961}, which states that if $w_i \neq 0$, $i = 1,
\ldots, m$, then \[ \Pr\left(\sum_{i = 1}^m w_i \zeta_i^2 > 0 \right) =
\frac{1}{2} + \frac{1}{\pi} \int_0^{+\infty} f(u) du, \] where
\begin{align*}
  f(u) &= \frac{\sin \theta(u)}{u \rho(u)}, \\
  \theta(u) &= \frac{1}{2} \sum_{i = 1}^m \arctan(w_iu), \\
  \rho(u) &= \prod_{i = 1}^m (1 + w_i^2u^2)^{1 / 4}.
\end{align*}
It is shown in (3.3) of \cite[p.~423]{imhof-1961} that
\begin{align}
  \nonumber
  \lim_{u \rightarrow 0} f(u) &= \frac{1}{2} \sum_{i = 1}^m w_i, \\
  \nonumber
  \lim_{u \rightarrow \infty} \theta(u) &= \frac{\pi}{4}
  \sum_{i = 1}^m \sgn w_i, \\
  \label{eq:dwcdftail}
  |\frac{1}{\pi} \int_u^{+\infty} f(u) du| & \leq
  \frac{1}{\pi} \frac{2}{m} u^{-m / 2} \prod_{i = 1}^m |w_i|^{-1 / 2}.
\end{align}

To achieve the result $\mathit{prob}$ and the accuracy $\mathit{eps}$
such that
\begin{equation} \label{eq:dwcdfprob}
  \mathit{prob} - \mathit{eps} \leq \Pr \left( \sum_{i = 1}^m w_i
  \zeta_i^2 > 0 \right) < \mathit{prob} + \mathit{eps}
\end{equation}
we put
\begin{equation} \label{eq:dwcdfupperlimit}
  u \geq (\frac{1}{\pi} \frac{2}{m} (\prod_{i = 1}^m |w_i|^{-1 / 2}) /
  (\mathit{eps} / 2))^{2 / m}
\end{equation}
so that by~(\ref{eq:dwcdftail}) we have \[ |\frac{1}{\pi}
\int_u^{+\infty} f(u) du| \leq \mathit{eps} / 2 \] and then calculate
\begin{equation} \label{eq:dwcdfintegral}
  \frac{1}{\pi} \int_0^u f(u) du
\end{equation}
by the Simpson method until two consecutive approximations differ no
more than $\mathit{eps} / 2$. To have four significant digits after the
comma, we put $\mathit{eps} = 0.000049$.

The function should not be used for low values of $n - k - 1$, say 1, 2,
3, as integration~(\ref{eq:dwcdfintegral}) fails. For $n - k - 1 = 1$ we
have
\begin{align*}
  \Pr(d_L < x) = 0 & \quad \mathrm{for} \quad x \leq \lambda_1, \\
  \Pr(d_L < x) = 1 & \quad \mathrm{for} \quad x > \lambda_1, \\
  \Pr(d_U < x) = 0 & \quad \mathrm{for} \quad x \leq \lambda_{1 + k}, \\
  \Pr(d_U < x) = 1 & \quad \mathrm{for} \quad x > \lambda_{1 + k}.
\end{align*}
For possible solution to this problem, see~\cite{farebrother-1984}. See
also~\cite{savin-white-1977}.
