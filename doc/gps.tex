\chapter{GPS}

Suppose that the Earth is a spheroid
\begin{equation}
  \label{eq:spheroid}
  \frac{x^2}{a^2} + \frac{y^2}{a^2} + \frac{z^2}{b^2} = 1, \quad a
  \geq b > 0,
\end{equation}
with its centre of symmetry in the origin of the Cartesian coordinate
system and axes overlapping axes of coordinate system.

Let $P$ be a point on the ellipsoid~\eqref{eq:spheroid}. Define
geographical coordinates: latitude and longitude as follows. Latitude
of $P$ is the angle between the normal to the ellipsoid at the point
$P$ and the plane $z = 0$ (see figure \ref{fig:spheroid_coordinates}).
Longitude of $P$ is the angle between the plane $y = 0$ and the plane
perpendicular to the plane $z = 0$ passing through the point $P$ and
the origin of the coordinate system.

\begin{figure}[ht]
\begin{center}
\resizebox{6cm}{!}{\includegraphics{1.pdf}}
\end{center}
\caption{Cartesian and geographic coordinates on the spheroid.}
\label{fig:spheroid_coordinates}
\end{figure}

Denote latitude by $-\frac{\pi}{2} \leq \varphi \leq \frac{\pi}{2}$,
longitude by $-\pi \leq \lambda \leq \pi$. The $X$ axis passes through
the point $(a, 0, 0)$ with latitude $0^\circ$ and longitude $0^\circ$
($\varphi = 0, \lambda = 0$), the $Y$ axis passes through the point
$(0, a, 0)$ with latitude $0^\circ$ and longitude $90^\circ$ east
($\varphi = 0, \lambda = \frac{\pi}{2}$) and the $Z$ axis passes
through the north pole $(0, 0, b)$ with latitude $90^\circ$ north
($\varphi = \frac{\pi}{2}, -\pi \leq \lambda \leq \pi$).

\section{Conversion of coordinates}

Let $P$ be a point on the ellipsoid~\eqref{eq:spheroid} with latitude
$0 < |\varphi| < \frac{\pi}{2}$ and longitude $-\pi \leq \lambda \leq
\pi$. We are looking for its coordinates $(x, y, z)$. As the
$Z$-coordinate depends only on $\varphi$, take a point $P'$ with
latitude $\varphi$ and longitude $0$, lying on the plane $y = 0$ (see
figure \ref{fig:plane_y_0}). The line passing through $P'$ and
inclined to the $x$-axis at the angle $\varphi$ is a normal to the
ellipse
\begin{equation} \label{eq:ellipseXZ}
  \frac{x^2}{a^2} + \frac{z^2}{b^2} = 1
\end{equation}
at a point $(x_0, z_0)$ and intersects the $x$-axis at the point
$(x_1, 0)$.

\begin{figure}[ht]
\begin{center}
\resizebox{6cm}{!}{\includegraphics{2.pdf}}
\end{center}
\caption{Position of the point $P'$ on the plane $y = 0$.}
\label{fig:plane_y_0}
\end{figure}

\noindent Hence it follows that
\begin{equation} \label{eq:first_eq_for_z}
  \tg \varphi = \frac{z_0}{x_0 - x_1}.
\end{equation}
This normal has the equation
\[
z_0x - (x_0 - x_1)z - z_0x_1 = 0.
\]
On the other hand, a normal to the ellipse \eqref{eq:ellipseXZ} at its
point $(x_0, z_0)$ has the equation \cite [pp.~215,~244]
{bronsztejn-siemiendiajew-musiol-muhlig-2004}
\[
a^2z_0x - b^2x_0z - (a^2 - b^2)x_0z_0 = 0.
\]
These two lines must overlap, so there must be
\[
\frac{z_0}{a^2z_0} = \frac{x_0 - x_1}{b^2x_0} = \frac{z_0x_1}{(a^2 -
  b^2)x_0z_0}
\]
and
\begin{equation} \label{eq:x1}
  x_0 - x_1 = \frac{b^2}{a^2}x_0, \quad x_1 = \frac{a^2 -
    b^2}{a^2}x_0.
\end{equation}
From \eqref{eq:ellipseXZ} we have
\begin{equation} \label{eq:x0_from_XZ}
  x_0 = \frac{a}{b}\sqrt{b^2 - z_0^2}.
\end{equation}
Using \eqref{eq:x1} and \eqref{eq:x0_from_XZ} in
\eqref{eq:first_eq_for_z} we obtain
\[
z_0 = p \frac{\sin \varphi}{\sqrt{1 - e^2\sin^2 \varphi}},
\]
where $p = b^2 / a$ \cite [p.~214]
{bronsztejn-siemiendiajew-musiol-muhlig-2004} and
\[
e = \frac{\sqrt{a^2 - b^2}}{a}
\]
is the eccentricity of the ellipse~\eqref{eq:ellipseXZ} \cite [p.~214]
{bronsztejn-siemiendiajew-musiol-muhlig-2004}.
From~\eqref{eq:x0_from_XZ} we obtain
\[
x_0 = a \frac{\cos \varphi}{\sqrt{1 - e^2\sin^2 \varphi}}.
\]
$X$ and $Y$ coordinates can be calculated from the parametric equation
of the circle $x^2 + y^2 = x_0^2$ for the angle $\lambda$: $x = x_0
\cos \lambda$, $y = x_0 \sin \lambda$.

The complete coordinates are
\begin{align*}
  x &= a \frac{\cos \varphi}{\sqrt{1 - e^2\sin^2 \varphi}} \cos \lambda \\
  y &= a \frac{\cos \varphi}{\sqrt{1 - e^2\sin^2 \varphi}} \sin \lambda \\
  z &= p \frac{\sin \varphi}{\sqrt{1 - e^2\sin^2 \varphi}}
\end{align*}
These formulas are valid also for $\varphi = 0$ and $\varphi = \pm
\frac{\pi}{2}$.

If a point lies on the normal with distance $h$ from the surface of
the ellipsoid, then the corresponding values of $z_0$ and $x_0$ should
be increased by $h \sin \varphi$ and $h \cos \varphi$, respectively.
Therefore, if a point is specified by coordinates $(\varphi, \lambda,
h)$, corresponding Cartesian coordinates are
\begin{align}
  \label{eq:x_coord}
  x &= \left( \frac{a}{\sqrt{1 - e^2\sin^2 \varphi}} + h \right) \cos
  \varphi \cos \lambda \\
  \label{eq:y_coord}
  y &= \left( \frac{a}{\sqrt{1 - e^2\sin^2 \varphi}} + h \right) \cos
  \varphi \sin \lambda \\
  \label{eq:z_coord}
  z &= \left( \frac{p}{\sqrt{1 - e^2\sin^2 \varphi}} + h \right) \sin
  \varphi
\end{align}

\section{Parameters of the spheroid}

We specify the spheroid of the Earth by two parameters\footnote{See
  \url{https://en.wikipedia.org/wiki/World_Geodetic_System}.}:
semi-major axis $a = 6378137.0$ metres and inverse flattening $1 / f =
298.257223563$, where $f = (a - b) / a$.
