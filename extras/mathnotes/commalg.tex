\chapter{Commutative algebra}
\setcounter{section}{-1}

This section is rewritten from \cite{eisenbud-2004}[section 0].

\section{Elamentary Definitions}
\subsection{Rings and Ideals}
A \textbf{ring} is an abelian group $R$ with an operation $(a, b)
\mapsto ab$ called \emph{multiplication} and an ``identity element''
1, satisfying, for all $a, b, c \in R$:

\begin{align*}
  a(bc) &= (ab)c && \text{(associativity)} \\
  a(b + c) &= ab + ac \\
  (b + c) a &= ba + ca && \text{(distributivy)} \\
  1a &= a1 = a && \text{(identity)}.
\end{align*}
A ring $R$ is \textbf{commutative} if, in addiotion, $ab = ba$ for all
$a, b \in R$.

\begin{exercise}
Show that $\mathcal{P}(\Omega)$ with addition $A \mathbin{\triangle} B
= (A \cup B) \setminus (A \cap B)$ and multiplication $A \cap B$ is a
commutative ring with multiplicative identity, in which each non-zero
element of its additive group is of order two.
\end{exercise}
\begin{proof}
  We use identities
  \begin{gather*}
    A \mathbin{\triangle} B = (A \cap B') \cup (A' \cap B), \\
    (A \cup B) \cap C = (A \cap C) \cup (B \cap C), \\
    (A \cup B) \cap (C \cup D) = (A \cap C) \cup (A \cap D) \cup (B
    \cap C) \cup (B \cap D).
  \end{gather*}
  To show associativity of addition $(A \mathbin{\triangle} B)
  \mathbin{\triangle} C = A \mathbin{\triangle} (B \mathbin{\triangle}
  C)$ we write both sides:
\end{proof}
