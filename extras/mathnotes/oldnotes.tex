\chapter{Old notes (until 2021)}

\begin{note}[Functions of multivariate random variables] \label{note:fmvrv}
Let $(X_1, \ldots, X_n)$ be a random vector with density $f(x_1,
\ldots, x_n)$. Let
\begin{align*}
U_1 &= g_1(X_1, \ldots, X_n) \\
& \ldots \\
U_n &= g_n(X_1, \ldots, X_n)
\end{align*}
be a one-to-one differentiable function with Jacobian determinant
different from 0 on $[a_1, b_1) \times \ldots \times [a_n, b_n)$. Let
\begin{align*}
X_1 &= h_1(U_1, \ldots, U_n) \\
& \ldots \\
X_n &= h_n(U_1, \ldots, U_n)
\end{align*}
be the inverse function and let
\[ J = \left|
\begin{array}{ccc}
  \frac{\partial x_1}{\partial u_1} &
  \ldots &
  \frac{\partial x_1}{\partial u_n} \\
  \hdotsfor{3} \\
  \frac{\partial x_n}{\partial u_1} &
  \ldots &
  \frac{\partial x_n}{\partial u_n} \\
\end{array}
\right|
\]
be its Jacobian determinant. Then the density of $(U_1, \ldots, U_n)$
is
\[
h(u_1, \ldots, u_n) = f[h_1(u_1, \ldots, u_n), \ldots, h_n(u_1,
  \ldots, u_n)]|J|.
\]
\end{note}

\begin{proof}
See \cite[\S2.9, p.~67]{fisz-1969}.
\end{proof}

\begin{note}
Let $X_1, \ldots, X_n$ be independent identically distributed random
variables with common density $f$. Let $(i_1, \ldots, i_n)$ be a
permutation of the set $\{1, \ldots, n\}$. Then $\Pr(X_{i_1} < X_{i_2}
< \ldots < X_{i_n}) = 1 / n!$.
\end{note}

\begin{proof}
Let us consider the transformation
\begin{align*}
U_1 &= X_{i_1} \\
& \ldots \\
U_n &= X_{i_n}
\end{align*}
The Jacobian matrix of this transformation is a permutation matrix,
whose $i$-th row contains 1 in the $i_j$-th column and zeros in the
remaining columns. The determinant of this matrix is equal to $\pm 1$.
By the note~\ref{note:fmvrv}, $(U_1, \ldots, U_n)$ has the density
\[
h(u_1, \ldots, u_n) = f(x_{i_1}) \ldots f(x_{i_n}) = f(x_1) \ldots f(x_n).
\]

Thus all permutations of $X_1, \ldots, X_n$ have the same density as
$X_1, \ldots, X_n$, ie. $f(x_1)\ldots f(x_n)$. Hence it follows that
for each permutation $(i_1, \ldots, i_n)$
\[
\Pr(X_{i_1} < X_{i_2} < \ldots < X_{i_n}) = \Pr(X_1 < X_2 < \ldots <
X_n) = 1 / n! \qedhere
\]
\end{proof}

\begin{note}[The smallest non-commutative group]
  Every group of order not greater than 5 is
  commutative.\footnote{\cite [p.~58] {lang-1984}.}
\end{note}

\begin{proof}
  Let a group $G$ contains the unit element $e$ and four elements $a,
  b, c, d$ such that
  \begin{align}
    ab &= c \label{eq:abc} \\
    ba &= d. \label{eq:bad}
  \end{align}

  There cannot be $a^{-1} = a$ because from \eqref{eq:abc} it would
  follow $ac = b$ and then it would have to be $ad = d$ ($ae = a$, $aa
  = e$, $ab = c$, $ac = b$, so the only possible result of $ad$ is
  $d$). There cannot be $a^{-1} = b$ because then it would be $ab =
  e$. There cannot be $a^{-1} = c$ because from \eqref{eq:abc} it
  would follow $cc = b$ and from \eqref{eq:bad} it would follow $dc =
  b$, so it would be $c = d$. There cannot be $a^{-1} = d$ because
  from~\eqref{eq:abc} it would follow $dc = b$ and from~\eqref{eq:bad}
  it would follow $dd = b$, so it would also be $c = d$. Therefore
  there must exist $f \in G - \{ e, a, b, c, d \}$ such that
  \begin{equation}
    af = fa = e. \label{eq:afe}
  \end{equation}

  Accept
  \begin{equation}
    bb = e. \label{eq:bbe}
  \end{equation}
  Now the group $G$ can be uniquely determined.

  From~\eqref{eq:abc} and~\eqref{eq:afe} we have $fc = b$,
  from~\eqref{eq:bad} and ~\eqref{eq:afe} we have $df = b$.
  From~\eqref{eq:abc} and~\eqref{eq:bbe} we have $cb = a$,
  from~\eqref{eq:bad} and ~\eqref{eq:bbe} we have $bd = a$. The
  multiplication table built so far is
  \begin{center}
  \begin{tabular}{l|llllll}
        & $e$ & $a$ & $b$ & $c$ & $d$ & $f$ \\ \hline
    $e$ & $e$ & $a$ & $b$ & $c$ & $d$ & $f$ \\
    $a$ & $a$ & $ $ & $c$ & $ $ & $ $ & $e$ \\
    $b$ & $b$ & $d$ & $e$ & $ $ & $a$ & $ $ \\
    $c$ & $c$ & $ $ & $a$ & $ $ & $ $ & $ $ \\
    $d$ & $d$ & $ $ & $ $ & $ $ & $ $ & $b$ \\
    $f$ & $f$ & $e$ & $ $ & $b$ & $ $ & $ $ \\
  \end{tabular}
  \end{center}
  We can see that there must be $bc = f$, $bf = c$, $db = f$, $fb =
  d$. The table is
  \begin{center}
  \begin{tabular}{l|llllll}
        & $e$ & $a$ & $b$ & $c$ & $d$ & $f$ \\ \hline
    $e$ & $e$ & $a$ & $b$ & $c$ & $d$ & $f$ \\
    $a$ & $a$ & $ $ & $c$ & $ $ & $ $ & $e$ \\
    $b$ & $b$ & $d$ & $e$ & $f$ & $a$ & $c$ \\
    $c$ & $c$ & $ $ & $a$ & $ $ & $ $ & $ $ \\
    $d$ & $d$ & $ $ & $f$ & $ $ & $ $ & $b$ \\
    $f$ & $f$ & $e$ & $d$ & $b$ & $ $ & $ $ \\
  \end{tabular}
  \end{center}
  From the table it follows that $ff = a$ and then that $fd = c$, $cf
  = d$. Since $ff = a$ and $f = a^{-1}$, then $a^{-1}a^{-1} = a$, so
  \begin{equation}
    aa = f. \label{eq:aaf}
  \end{equation}
  From~\eqref{eq:abc} and~\eqref{eq:aaf}:
  \begin{equation} \nonumber
    ab = c \Rightarrow aab = ac \Rightarrow ac = fb \Rightarrow ac = d.
  \end{equation}
  From~\eqref{eq:bad} and~\eqref{eq:aaf}:
  \begin{equation} \nonumber
    ba = d \Rightarrow baa = da \Rightarrow da = bf \Rightarrow da = c.
  \end{equation}
  The table becomes
  \begin{center}
  \begin{tabular}{l|llllll}
        & $e$ & $a$ & $b$ & $c$ & $d$ & $f$ \\ \hline
    $e$ & $e$ & $a$ & $b$ & $c$ & $d$ & $f$ \\
    $a$ & $a$ & $f$ & $c$ & $d$ & $ $ & $e$ \\
    $b$ & $b$ & $d$ & $e$ & $f$ & $a$ & $c$ \\
    $c$ & $c$ & $ $ & $a$ & $ $ & $ $ & $d$ \\
    $d$ & $d$ & $c$ & $f$ & $ $ & $ $ & $b$ \\
    $f$ & $f$ & $e$ & $d$ & $b$ & $c$ & $a$ \\
  \end{tabular}
  \end{center}
  We complete $ad = b$ and $ca = b$. Multiplying~\eqref{eq:abc}
  and~\eqref{eq:bad} we have
  \begin{equation} \nonumber
    cd = abba = aa = f, \quad dc = baab = bfb = cb = a.
  \end{equation}
  Completing we have $cc = e$ and $dd = e$. The final table is
  \begin{center}
  \begin{tabular}{l|llllll}
        & $e$ & $a$ & $b$ & $c$ & $d$ & $f$ \\ \hline
    $e$ & $e$ & $a$ & $b$ & $c$ & $d$ & $f$ \\
    $a$ & $a$ & $f$ & $c$ & $d$ & $b$ & $e$ \\
    $b$ & $b$ & $d$ & $e$ & $f$ & $a$ & $c$ \\
    $c$ & $c$ & $b$ & $a$ & $e$ & $f$ & $d$ \\
    $d$ & $d$ & $c$ & $f$ & $a$ & $e$ & $b$ \\
    $f$ & $f$ & $e$ & $d$ & $b$ & $c$ & $a$ \\
  \end{tabular}
  \end{center}
\end{proof}

\begin{note}[Commutative non-associative ,,group'']
  The set $\{0, 1, 2, 3\}$ with the sum $x + y = |x - y|$ meets the
  requirements of commutative group without associativity.
  \footnote{\url{https://math.stackexchange.com/questions/1622348}.}
\end{note}

\begin{proof}
  The table of addition is
  \begin{center}
  \begin{tabular}{l|llllll}
        & $0$ & $1$ & $2$ & $3$ \\ \hline
    $0$ & $0$ & $1$ & $2$ & $3$ \\
    $1$ & $1$ & $0$ & $1$ & $2$ \\
    $2$ & $2$ & $1$ & $0$ & $1$ \\
    $3$ & $3$ & $2$ & $1$ & $0$ \\
  \end{tabular}
  \end{center}

  The zero element is 0, addition is commutative, for each element
  $x$, $-x = x$. However, $(1 + 2) + 3 = 1 + 3 = 2$ and $1 + (2 + 3) =
  1 + 1 = 0$.

  The set $\realn_+$ with the sum $x + y = |x - y|$ has the same
  properties.
\end{proof}
